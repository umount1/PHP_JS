
\documentclass[pdflatex,a4paper,11pt]{scrartcl}
\author{J. Schönbohm}
\date{\today}
\usepackage[ngerman]{babel}
\usepackage[utf8]{inputenc}
\usepackage[T1]{fontenc}
%\usepackage{bookman}
\usepackage{graphicx} % Einbinden vorhandener Bilder (jpg, png, pdf)
\usepackage{listings}
\usepackage{color}
\usepackage{amsmath}
\usepackage{amsfonts}
\begin{document}
\section*{Aufgabenblatt 1}
\lstset{
   frameround         = fttt,
   language           = C++,
   extendedchars      = true,
   basicstyle         = \ttfamily\small,
   backgroundcolor    = \color[gray]{0.95},
   keywordstyle		  = \color{blue}\bfseries,
   identifierstyle    = ,
   commentstyle		  = \color{green},
   morecomment		  = [is]{/*}{*/},
   stringstyle        = \color{red},
%   numbers            = left,
%   numberstyle        = \tiny,
   frame              = single,
   framexleftmargin   = 5pt,
   rulesep            = 5pt,
   framexrightmargin  = 5pt,
   framexbottommargin = 5pt,
   framextopmargin    = 5pt,
   xleftmargin        = 5pt,
   xrightmargin       = 5pt,
   framerule          = 0pt,
   tabsize            = 3,
   breaklines         = true,
   showstringspaces	  = false,
   captionpos         = b}
\subsection*{1.1 Quersumme} 
Schreiben Sie ein Programm, das 
die Quersumme einer eingegebenen, nat\"urlichen Zahl $n$ rekursiv berechnet. Die rekursive Funktion $q(n)$ folgt folgender Vorschrift:
$$q(n)=\left\{ 
\begin{array}{cl}
n,&n<10\\
q(\lfloor n/10\rfloor ) + n\ mod\ 10, &n\ge 10
\end{array}
\right.$$
\subsection*{1.2 Gebiete} 
Wenn man ein konvexes Gebiet $M$ durch eine Gerade teilt entstehen 2 Gebiete. Teilt man $M$ durch 2 Geraden, so entstehen maximal 4 Gebiete. \newline
Schreiben Sie ein Programm, das 
die maximale Anzahl an Gebieten berechnet, die man erh\"alt, wenn man $M$ durch $n$ Geraden teilt. Die Funktionsvorschrift f\"ur $f(n)$ lautet:

$$q(n)=\left\{ 
\begin{array}{cl}
2,&n=1\\
f(n-1) + n, &n\ge 2
\end{array}
\right.$$
\subsection*{1.3 Schnelles Potenzieren} 
Schreiben Sie ein Programm, f\"ur schnelles Potenzieren $x^y$ mit $x\in\mathbb{R}$ und $y\in\mathbb{N}$:

$$x^y=\left\{ 
\begin{array}{cl}
x^\frac{y}{2}\cdot x^\frac{y}{2},&\mbox{wenn }y\mbox{ gerade}\\
x\cdot x^{\lfloor\frac{y}{2}\rfloor}\cdot x^{\lfloor\frac{y}{2}\rfloor} , &\mbox{ sonst}
\end{array}
\right.$$
\end{document}